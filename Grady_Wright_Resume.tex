\documentclass[11pt, letterpaper]{article}
\usepackage{geometry}
\usepackage{multicol}
\usepackage{enumitem}
% \usepackage{showframe}
\usepackage{titlesec}
\usepackage{fontawesome}
\usepackage{calc}
\usepackage[none]{hyphenat}
\pagenumbering{gobble}
\geometry{left=1.2cm,right=1.2cm,top=1.5cm,bottom=1.25cm}
\titlespacing*{\section}{0pt}{*2}{*1}
\titleformat{\section}    
       {\normalfont\fontsize{17}{19}\bfseries}{\thesection}{1em}{}
\titlespacing*{\subsection}{0pt}{*2}{0pt}
\titleformat{\subsection}    
       {\normalfont\fontfamily{phv}\fontsize{12}{17}\bfseries}{\thesubsection}{1em}{}
\newenvironment{desItemize}
{ \begin{itemize}[leftmargin=*, topsep=1pt]
    \setlength{\itemsep}{0pt}
    \setlength{\parskip}{0pt}
    \setlength{\parsep}{0pt}
    \small     }
{ \end{itemize}                  } 

\newcommand{\columnRule}{
    \noindent\hspace{-0.5\columnsep}\rule{\columnwidth + \columnsep/2}{1pt}\noindent
}

\newenvironment{leftSection}[1]
{\section*{#1}}
{\rule{\columnwidth + \columnsep/2}{1pt}}

\newenvironment{rightSection}[1]
{\section*{#1}}
{\columnRule}

\setlength{\columnsep}{1cm}

\begin{document}
{\noindent\hspace{-1mm}\Huge\textbf{Grady Wright}\par}
{\vspace{2mm}\noindent\Large\textbf{Computer Engineer}\par}
{\vspace{2mm}\noindent\small\faEnvelope\ gowright98@gmail.com\hfill \faExternalLinkSquare\ www.grady-wright.com\hfill \faLinkedin\ linkedin.com/in/grady-wright\hfill \faGithub\ github.com/littleg13}
\normalsize\vspace{3mm}
\begin{multicols*}{2}[]
    \raggedcolumns
\begin{leftSection}{Education}
    \subsection*{University of Kansas}
    Bachelor of Science\hfill\faCalendar\ May 2020\\
    Major: Computer Engineering \hfill GPA: 3.95
\end{leftSection}
\begin{leftSection}{Experience}
    \subsection*{Software Engineer}
    \textbf{SpaceX}\hfill
    \faCalendar\ 1/2021 - current
    \begin{desItemize}
        \item	Create large scale automated UI testing framework
        \item	Write Python to implement automated writing and executing of UI tests
        \item	Implement features and fix bugs for Garmin chart plotters using C
        \item	Work with image comparison algorithms to implement UI test validation
    \end{desItemize}
    \subsection*{Software Engineering Intern}
    \textbf{Garmin Ltd}\hfill
    \faCalendar\ 5/2019 - 8/2019
    \begin{desItemize}
        \item	Create large scale automated UI testing framework
        \item	Write Python to implement automated writing and executing of UI tests
        \item	Implement features and fix bugs for Garmin chart plotters using C
        \item	Work with image comparison algorithms to implement UI test validation
    \end{desItemize}
    \subsection*{Undergraduate Research Assistant}
    \textbf{The University of Kansas}\hfill
    \faCalendar\ 1/2019 - 6/2019
    \begin{desItemize}
        \item	Assist in development of real-time raytracing simulation of communication networks
        \item	Develop CUDA C++ for simulation of raytracing
        \item	Write C++ for visualization of simulation using OpenGL
    \end{desItemize}
\end{leftSection}
\begin{leftSection}{Involvement}
    \subsection*{Tau Beta Pi}
    \textbf{Engineering Honor Society}
    \begin{desItemize}
    \item Active member through volunteering, tutoring, and test preparation
    \end{desItemize}

    \subsection*{Eta Kappa Nu}
    \textbf{IEEE Honor Society}
    \begin{desItemize}
    \item Active member through volunteering and coursework assistance
    \end{desItemize}
\end{leftSection}
\columnbreak
\begin{rightSection}{Skills}
    \begin{desItemize}
    \item	Proficient with Python, C++, C\#, C, HTML/CSS/JS, MATLAB
    \item	Proficient using OpenGL and CUDA
    \item	Experience with ray-marching and signed distance fields
    \item	Extensive experience with automated UI and regression testing
    \item	3D modeling experience using AutoDesk Maya 2018, Blender, AutoDesk Inventor 2018
    \end{desItemize}
\end{rightSection}
\begin{rightSection}{Honors and Awards}
    \subsection*{First Place HackKU 2020}
    \textbf{The University of Kansas}
    \begin{desItemize}
    \item 1st place in the FinTech track of KU’s annual hackathon for development of peer-to-peer money transfer system
    \end{desItemize}
    \subsection*{Undergraduate Achievement Award}
    \textbf{Eta Kappa Nu}
    \begin{desItemize}
    \item Awarded to top students in EECS 221 Electromagnetics
    \end{desItemize}
\end{rightSection}
\begin{rightSection}{Projects}
    \subsection*{Twitch Overlay}
    \textbf{Computer Graphics}
    \begin{desItemize}
    \item OpenGL rendered mesh particle system that updates with information gathered from the Twitch API
    \item Communicates with Twitch API using HTTP/Websocket protocols written with Winsock 
    \item Supports separate build for ray-marched SDFs allowing for soft-shadows and reflections
    \end{desItemize}
    \subsection*{Fluid Simulation}
    \textbf{Computer Graphics}
    \begin{desItemize}
    \item SPH fluid simulation built entirely in OpenGL
    \item Utilizes OpenGL compute shaders to apply particle physics and update render position
    \item Optimization is done using spatial hashing on particle positions
    \end{desItemize}
    \subsection*{Mooxter}
    \textbf{HackKU 2020}
    \begin{desItemize}
    \item Peer-to-peer payment transfer system to utilizing everyday applications, such as Discord, Slack, Twitter
    \item Allows for XRP to be sent utilizing the Xpring API
    \item Utilizes Kubernetes to easily add interfaces and scale
    \end{desItemize}
\end{rightSection}
\end{multicols*}
\end{document}